\documentclass[twocolumn]{article}

\usepackage{style}

\renewcommand{\thesection}{}

\begin{document}
  \title{inf3410 - Formulas}
  \author{Robin A. T. Pedersen}
  \maketitle

  Caution! Some formulas may have preconditions not mentioned here.

  \section{Capitilization Convention}
    $$\text{ex:}\quad i_D = I_D + i_d$$
    \begin{align}
      v_{in} &\implies \text{has only AC (small signal)} \\
      v_{IN} &\implies \text{has both DC + AC} \\
      V_{IN} &\implies \text{is a pure DC signal} \\
      V_{in} &\implies \text{wtf is this?}
    \end{align}

  \section{CMOS FET}
    TODO cheatsheet of nMOS/pMOS regions, equations, iD(vSD).
    TODO chapter 5.1 - 5.2.

    \paragraph{Resistance and Transconductance} \hfill \\
      Triode/Linear region (\emph{very} small $v_{DS}$):
        $$r_{DS} = \frac{1}{g_{DS}}$$
        $$g_{DS} = k_n v_{OV}$$
        $$k_n = k_n' \cdot \frac{W}{L}$$
        $$v_{OV} = v_{GS} - V_t$$
      Triode region ($v_{DS} < v_{OV}$):
        $$g_m = \text{slope?}
          = ?\frac{\delta i_D}{\delta v_{DS}}
          = ? k_n (v_{OV} - v_{DS})
          $$
      Active region:
        $$g_m = \frac{\delta I_D}{\delta V_{GS}} = \frac{2 I_D}{V_{ov}}$$
        $$r_o = \frac{1}{\lambda I_D}$$

    \paragraph{pMOS} \hfill \\
      Cutoff (no channel):
        $$v_{SG} < |V_{tp}| \quad\implies\quad i_D \approx 0$$
      Overdrive Voltage:
        $$|v_{OV}| = v_{SG} - |V_{tp}|$$
      Triode Region:
        $$v_{DG} > |V_{tp}| \quad\Leftrightarrow\quad v_{SD} < |v_{OV}|$$
      Triode Current:
        $$i_D = k_p' \left( \frac{W}{L} \right)
                     \left( |v_{OV}| - \frac{1}{2} v_{SD} \right) v_{SD}$$
      Saturation Region:
        $$v_{DG} \leq |V_{tp}| \quad\Leftrightarrow\quad v_{SD} \geq |v_{OV}|$$
      Saturation Current:
        $$i_D = \frac{1}{2} k_p' \left( \frac{W}{L} \right) v_{OV}^2$$
    \paragraph{nMOS} \hfill \\
      It's flipped of pMOS $\ddot\smile$. Much simpler! Haha.

      Very low triode region:
        $$i_D = [k_n v_{OV}] v_{DS}$$
    \paragraph{EKV model} \hfill \\
      $$i_D = i_F - i_R$$
      $$i_{F(R)} = I_S \cdot
                   \ln \left[ 1 + e^{\frac{v_G - V_{tn} - nv_{S(D)}}{nV_T}}
                       \right] ^2
                   (1 + \lambda v_{DS})$$
      TODO $I_S$?
    \paragraph{Channel Length Modulation} \hfill \\
      TODO $... \cdot (1 + \lambda v_{DS}$?

      $$V_A = \frac{1}{\lambda}$$
      $$V_A = V_A' \cdot L?$$

  \section{Thevenin Circuits}
    \paragraph{Thevenin Resistance} \hfill \\
      Short-circuit all voltage sources.
      Then, calculate the resistance $R_{Th}$ from the output terminal relative
      to ground.
      You calculate parallell and series resistors just like you normally
     calculate resistance.
    \paragraph{Thevenin Voltage} \hfill \\
      Short-circuit all voltage sources, except one.
      Then calculate the voltage on the output terminal.
      Do this for all of the voltage sources.
      The thevenin voltage $V_{Th}$ is the sum of all the voltages you found.
      $$V_{Th} = V_1 + V_2 + ...$$
    \paragraph{Intermediate Thevenin Voltages} \hfill \\
      Find a simple, innermost, small part of the circuit.
      Calculate the thevenin voltage for this node (usually a simple voltage
      divider).
      Use that to find the voltage of the next node.
      NB! For the next node, be carefull to use thevenin resistances in
      your calculations.

      TODO write as enumeration.

  \section{Common Mode Rejection Ratio (CMRR)}
    $$v_{o,max} = A_d \cdot v_{id,max} + A_{cm} \cdot v_{cm,max}$$
    $$A_{cm} = \frac{A_d}{\text{CMRR}}$$
    $$A_{cm,dB} = A_{d,dB} - \text{CMRR}_{dB}$$
    $$V_{cm,max} = \left[\max(V_+) + \max(V_-) \right] \cdot \frac{1}{2}$$
    $$V_{cm,min} = \left[\min(V_+) + \min(V_-) \right] \cdot \frac{1}{2}$$

  \section{Current Mirror}
    From $V_{DD}$ through $R$ we get to the nMOS.
    $$v_{DS} = V_{DD} - V_R$$
    The D is connected to the G.
    $$v_{DS} = v_{GS}$$
    This forces the nMOS to always saturate.
    $$I_D = \frac{1}{2} k_n v_{OV}^2 = \frac{V_{DD} - V_{GS}}{R} = I_{REF}$$
    For any following nMOS connected Gate-To-Gate, this will get the same
    current as long as it is in saturation (BIG enough $v_{DS}$).

    \paragraph{Relation to width and length} \hfill \\
    The two transistors receives to same gate voltage.
    $$I_1 = I_{REF} = \frac{1}{1} k_n' \frac{W_1}{L_1} v_{OV}^1
      ,\quad
      I_2 = \frac{1}{2} k_n' \frac{W_2}{L_2} v_{OV}^2
      $$
    $$\implies \frac{I_2}{I_{REF}} = \frac{(W/L)_2}{(W/L)_1}$$

  \section{Intrinsic Gain}
    The maximum attainable gain in a CS amp.
    $$A_0 = \text{Intrinsic Gain} = g_m r_o = 2V_A / V_{OV}$$
    Combined with $i_{D,sat}$.
    $$A_0 = \frac{V_A' \sqrt{2 k_n' WL}}{\sqrt{i_D}}$$

    \paragraph{Unconventional Gain Stages} \hfill \\
    Find gain from small signal equivalent circuit.
    E.g.:
    $$V_o = R_{tot} \cdot (-g_m \cdot V_i)$$

  \section{Common Source Amp}
    $$A = g_m \cdot (r_o || R_D)$$

    TODO more info?

  \section{Current Source}
    From $V_{DD}$, through a resistor, through a nMOS to $GND$.
    The Gate and Drain are connected.
    The Gate can be connected to the Gate of another nMOS, to clone the current.

    \paragraph{Current Source Load} \hfill \\
      In a CS amp, replace $R_D$ with a pMOS (the current source).

  \section{Bode Plot}
    $(\tau s + 1)$ is $0dB$ until $\omega_0 = \frac{1}{\tau}$, then falls/grows
    by 20dB/dec.
    $$\frac{1}{\tau s + 1} = \text{Flat, then falls} = \text{Low-Pass Filter}$$
    $$\frac{1}{(\tau s + 1)^2} = \text{Flat, then falls (by 40dB/dec!)}$$
    $$\frac{\tau s + 1}{1} = \text{Flat, then grows}$$

    $(\tau s)$ always falls/grows, crosses $0dB$ at $\omega_o = \frac{1}{\tau}$.
    $$\frac{1}{\tau s} = \text{Always falls}$$
    $$\frac{\tau s}{1} = \text{Always grows}$$

    $\frac{\tau s - 1}{1}$:
    Grows upwards like $\frac{\tau s + 1}{1}$, but has negative phase like
    $\frac{1}{\tau s + 1}$.

    Amplification:
    $$\frac{100}{1} = \text{40dB amplification everywhere}$$

  \section{Logarithm}
    $$20 \cdot \log_{10}(A) = A_{dB}$$
    $$10^{\frac{A_{dB}}{20}} = A$$

  \section{Weak Inversion}
    From an exam:
    $$I_d = I_S \cdot e^{\frac{V_{gs}}{n V_T}}$$

    Active region:
    $$i_D = I_S \cdot e^{\frac{v_G - V_{tn} - n v_S}{nV_T}}
            \cdot (1 + \lambda v_{DS})$$

    TODO more info?

  \section{Open-Circuit Time-Constant Analysis}
    Short-circuit all voltage sources (TODO signal sources?).

    \paragraph{Capacitor No.1 ($C_1$)} \hfill \\
      Replace $C_1$ with a voltage source $V_X$.
      Rip out all other caps (open-circuit).
  
      Find the resistance $R_X$ \emph{seen by} your cap $C_1$.
  
      Congratz. You found the first time constant.
      $$\tau_1 = R_{X1} \cdot C_1$$

    \paragraph{Capacitor No.2} \hfill \\
      Repeat the same procedure for $C_2$.

    \paragraph{Result} \hfill \\
      $$\omega_0 = \frac{1}{\tau_1 + \tau_2 + ... + \tau_n}$$

  \section{Operational Transconductance Amplifier (OTA)}
    $$A = G_{MA} \cdot r_{out}$$

  TODO alphabetical order of sections?
\end{document}
